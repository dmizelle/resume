\documentclass[9pt]{developercv}

\usepackage{pagecolor}

\begin{document}
\pagecolor{black}
\color{white}
\thispagestyle{firstpage}

\begin{minipage}[t]{0.45\textwidth} % 45% of the page width for name
	\vspace{-\baselineskip} % Required for vertically aligning minipages
	
	% If your name is very short, use just one of the lines below
	% If your name is very long, reduce the font size or make the minipage wider and reduce the others proportionately
	\colorbox{black}{{\HUGE\textcolor{white}{\textbf{\MakeUppercase{Devon}}}}} % First name
	
	\colorbox{black}{{\HUGE\textcolor{white}{\textbf{\MakeUppercase{Mizelle}}}}} % Last name
	
	\vspace{6pt}
	
	{\huge Staff Site Reliability Engineer} % Career or current job title
\end{minipage}
\begin{minipage}[t]{0.275\textwidth} % 27.5% of the page width for the first row of icons
	\vspace{-\baselineskip} % Required for vertically aligning minipages
	
	% The first parameter is the FontAwesome icon name, the second is the box size and the third is the text
	% Other icons can be found by referring to fontawesome.pdf (supplied with the template) and using the word after \fa in the command for the icon you want
	\icon{MapMarker}{12}{Brooklyn, NY}\\
	\icon{Phone}{12}{+1 757 753 1316}\\
	\icon{At}{12}{\href{mailto:dev+cv@devon.so}{dev@devon.so}}\\
\end{minipage}
\begin{minipage}[t]{0.275\textwidth} % 27.5% of the page width for the second row of icons
	\vspace{-\baselineskip} % Required for vertically aligning minipages
	
	% The first parameter is the FontAwesome icon name, the second is the box size and the third is the text
	% Other icons can be found by referring to fontawesome.pdf (supplied with the template) and using the word after \fa in the command for the icon you want
	% \icon{Globe}{12}{\href{https://alyx.vance.me}{alyx.vance.me}}\\
	\icon{Github}{12}{\href{https://github.com/dmizelle}{github.com/dmizelle}}\\
	\icon{Key}{12}{\href{https://keyserver.ubuntu.com/pks/lookup?search=0xE4A6AB2D&fingerprint=on&op=index}0xE4A6AB2D}\\
	\icon{Hashtag}{12}{\_dev on irc.libera.chat}\\
\end{minipage}

\vspace{0.5cm}

%----------------------------------------------------------------------------------------
%	INTRODUCTION, SKILLS AND TECHNOLOGIES
%----------------------------------------------------------------------------------------

\cvsect{Who Am I?}

\begin{minipage}[t]{0.5\textwidth} % 40% of the page width for the introduction text
	\vspace{-\baselineskip} % Required for vertically aligning minipages
	An SRE with over ten years of experience in the technology space
	with a passion for building platforms targeting both developer
	happiness and hardcore reliability. Developer happiness is paramount as an SRE.\\

	Experience with working on global systems providing infrastructure
	for the internet as a whole and scaling systems to meet customer demand.\\

	Additional passions include monitoring, observability, esoteric kernel
	inner workings, and creative solutions to tough problems.
\end{minipage}
\hfill % Whitespace between
\begin{minipage}[t]{0.2\textwidth} % 50% of the page for the skills bar chart
	\vspace{-\baselineskip} % Required for vertically aligning minipages
	\icon{Dharmachakra}{12}{Kubernetes}\\
	\icon{Aws}{12}{AWS}\\
	\icon{Database}{12}{PostgreSQL}\\
	\icon{Linux}{12}{Linux}\\
	\icon{Freebsd}{12}{*BSD}\\
\end{minipage}
\begin{minipage}[t]{0.2\textwidth}
	\vspace{-\baselineskip} % Required for vertically aligning minipages
	\icon{Globe}{12}{Terraform}\\
	\icon{Fire}{12}{Prometheus}\\
	\icon{Google}{12}{Golang}\\
	\icon{Python}{12}{Python}\\
	\icon{SpaceShuttle}{12}{Ansible}\\
\end{minipage}

%----------------------------------------------------------------------------------------
%	EXPERIENCE
%----------------------------------------------------------------------------------------

\cvsect{Experience}

\begin{entrylist}
	\entry
		{2018 -- }
		{Staff Site Reliability Engineer}
		{Peloton Interactive}
		{
		  \begin{itemize}
			\item Launched and spearheaded the Kubernetes (``Compute'') team which moved our monolithic code base from Chef-configured static EC2 instances onto a scalable and flexible cloud-native platform. Later on, this allowed developers to move out of the 8 year old Python code base and begin developing their own micro-services.
			\item Wrote and maintained a Terraform code base to deploy Kubernetes clusters (EKS) and automatically scaling node groups, with operators included.
			\item Wrote Captain Ahab, a tool for developers to create ephemeral deployments of their Pull Requests for manual testing and collaborating with code reviewers.
			\item Developed multiple operators in Golang to automate tasks on the platform such as automated upgrades of other operators and automation around spinning up new projects.
			\item Deployed ArgoCD after evaluating other solutions (such as Spinnaker) to automate deploying of Kubernetes applications for developers.
			\item Grew the Compute team from 1 to 8 total team members. Mentored and trained them to bring them up to speed quickly and begin contributing.
			\item Provided on-the-ground guidance and planning for multiple feature launches including, but not limited to:
				  \begin{itemize}
					\item Next Generation On-Demand Leaderboard, which deprecated the archaic system and moved to a Kotlin-backed micro-service that scaled to meet the demand of customers replaying on-demand content.
					\item Tags, a feature project that customers join a team to participate and compete together.
					\item Guide, our new computer vision strength product which required deep learning instances.
				  \end{itemize}
			\item Acted as a product owner and technical lead of large Kubernetes clusters of over 5,000 nodes with a total of 50,000 pods.
			\item Gave numerous internal talks and acted as a subject matter expert on:
				  \begin{itemize}
					\item Service Meshes and Container Network Interfaces (Istio, Cillium)
					\item Building resilient distributed systems
					\item Launching new micro-services from ``zero'' to Production.
					\item Kubernetes and Linux kernel internals
				  \end{itemize}
		  \end{itemize}
		}
	\entry
		{2013 -- 2018}
		{Database Systems Engineer}
		{American Registry for Internet Numbers}
		{
		  \begin{itemize}
			\item Managed Systems for ARIN that supported the global IPv6 architecture and internet routing registry functions.
			\item Consolidated multiple monitoring systems into a highly-available multi-site Sensu deployment that put monitoring into the hands of software developers.
			\item Oversaw and participated in the migration of the core registry database from Oracle to PostgreSQL.
			\item Developed and maintained an in-house high-availability PostgreSQL cluster that implemented automatic fail-over and fencing.
			\item Built, maintained, and supported a multi-master Puppet infrastructure and modules spanning over the geographical area of the United States and Canada.
			\item Spearheaded the move to Ansible, deployed Continuous Integration, and acted as a subject matter expert on it for the company.
			\item Designed and maintained a multi-site ELK (Elasticsearch, Logstash, Kibana) stack that provided alerting for critical events.
			\item Deployed an automatic DNSSEC signer-in-the-middle product to ensure proper signing of corporate ARIN zones and reverse IPv4/IPv6 delegations.
			\item Gave an opening talk at PGConfNYC 2014 about our migration to PostgreSQL, the solutions we used, and our lessons learned.
		  \end{itemize}
		  }
	\entry
		{2011 -- 2013}
		{Systems Architect I}
		{InMotion Hosting}
		{
		  \begin{itemize}
			\item Managed a system supporting >2,500 servers, hosting upwards of 500,000 websites, and mentored junior employees.
			\item Developed a Resource Usage control system built around the Linux cgroups functionality which promoted stability and finer-grain control for actions on the server.
			\item Implemented Varnish on the Corporate website which increased the maximum connections per second by eight-fold and provided the ability for load balancing and high availability.
			\item Developed documentation for the implementation of high performance optimizations such as Varnish, memcached, FastCGI, nginx, and APC to allow customers to sustain higher traffic spikes without affecting Quality of Service for other customers.
			\item Wrote and deployed numerous Nagios plugins to provide further introspection into the activities on our network and server fleet.
			\item Built a visibility dashboard in Ruby for the department that provide a quick ’heads-up display’ of Nagios criticals, network throughput, and security violations.
		  \end{itemize}
		}
\end{entrylist}

%----------------------------------------------------------------------------------------
%	EDUCATION
%----------------------------------------------------------------------------------------

\cvsect{Education}

\begin{entrylist}
	\entry
		{2010 -- 2012}
		{B.S. Business Administration, Network Engineering}
		{Old Dominion University}
		{}
	\entry
		{2008 -- 2010}
		{A.S. Business Administration}
		{Tidewater Community College}
		{}
\end{entrylist}

%----------------------------------------------------------------------------------------
%	ADDITIONAL INFORMATION
%----------------------------------------------------------------------------------------

\begin{minipage}[t]{0.4\textwidth}
	\vspace{-\baselineskip} % Required for vertically aligning minipages
	
	\cvsect{Related Hobbies}

	\begin{itemize}
		\item Open source contributions to numerous projects including Kustomize, ArgoCD, etc.
		\item Embedded board hacking
		\item 3D Printing
		\item NixOS packaging (NUR)
	\end{itemize}
\end{minipage}
%\hfill
\begin{minipage}[t]{0.5\textwidth}
	\vspace{-\baselineskip} % Required for vertically aligning minipages
	
	\cvsect{Mentoring}

	\begin{itemize}
		\item Provided mock interviewing guidance to recent graduates at WiTNY: Women in Technology New York
		\item Consistent mentoring to individuals looking to break into the tech industry
	\end{itemize}
\end{minipage}

%----------------------------------------------------------------------------------------

\end{document}
